% Options for packages loaded elsewhere
\PassOptionsToPackage{unicode}{hyperref}
\PassOptionsToPackage{hyphens}{url}
%
\documentclass[
]{article}
\usepackage{amsmath,amssymb}
\usepackage{iftex}
\ifPDFTeX
  \usepackage[T1]{fontenc}
  \usepackage[utf8]{inputenc}
  \usepackage{textcomp} % provide euro and other symbols
\else % if luatex or xetex
  \usepackage{unicode-math} % this also loads fontspec
  \defaultfontfeatures{Scale=MatchLowercase}
  \defaultfontfeatures[\rmfamily]{Ligatures=TeX,Scale=1}
\fi
\usepackage{lmodern}
\ifPDFTeX\else
  % xetex/luatex font selection
\fi
% Use upquote if available, for straight quotes in verbatim environments
\IfFileExists{upquote.sty}{\usepackage{upquote}}{}
\IfFileExists{microtype.sty}{% use microtype if available
  \usepackage[]{microtype}
  \UseMicrotypeSet[protrusion]{basicmath} % disable protrusion for tt fonts
}{}
\makeatletter
\@ifundefined{KOMAClassName}{% if non-KOMA class
  \IfFileExists{parskip.sty}{%
    \usepackage{parskip}
  }{% else
    \setlength{\parindent}{0pt}
    \setlength{\parskip}{6pt plus 2pt minus 1pt}}
}{% if KOMA class
  \KOMAoptions{parskip=half}}
\makeatother
\usepackage{xcolor}
\usepackage[margin=1in]{geometry}
\usepackage{color}
\usepackage{fancyvrb}
\newcommand{\VerbBar}{|}
\newcommand{\VERB}{\Verb[commandchars=\\\{\}]}
\DefineVerbatimEnvironment{Highlighting}{Verbatim}{commandchars=\\\{\}}
% Add ',fontsize=\small' for more characters per line
\usepackage{framed}
\definecolor{shadecolor}{RGB}{248,248,248}
\newenvironment{Shaded}{\begin{snugshade}}{\end{snugshade}}
\newcommand{\AlertTok}[1]{\textcolor[rgb]{0.94,0.16,0.16}{#1}}
\newcommand{\AnnotationTok}[1]{\textcolor[rgb]{0.56,0.35,0.01}{\textbf{\textit{#1}}}}
\newcommand{\AttributeTok}[1]{\textcolor[rgb]{0.13,0.29,0.53}{#1}}
\newcommand{\BaseNTok}[1]{\textcolor[rgb]{0.00,0.00,0.81}{#1}}
\newcommand{\BuiltInTok}[1]{#1}
\newcommand{\CharTok}[1]{\textcolor[rgb]{0.31,0.60,0.02}{#1}}
\newcommand{\CommentTok}[1]{\textcolor[rgb]{0.56,0.35,0.01}{\textit{#1}}}
\newcommand{\CommentVarTok}[1]{\textcolor[rgb]{0.56,0.35,0.01}{\textbf{\textit{#1}}}}
\newcommand{\ConstantTok}[1]{\textcolor[rgb]{0.56,0.35,0.01}{#1}}
\newcommand{\ControlFlowTok}[1]{\textcolor[rgb]{0.13,0.29,0.53}{\textbf{#1}}}
\newcommand{\DataTypeTok}[1]{\textcolor[rgb]{0.13,0.29,0.53}{#1}}
\newcommand{\DecValTok}[1]{\textcolor[rgb]{0.00,0.00,0.81}{#1}}
\newcommand{\DocumentationTok}[1]{\textcolor[rgb]{0.56,0.35,0.01}{\textbf{\textit{#1}}}}
\newcommand{\ErrorTok}[1]{\textcolor[rgb]{0.64,0.00,0.00}{\textbf{#1}}}
\newcommand{\ExtensionTok}[1]{#1}
\newcommand{\FloatTok}[1]{\textcolor[rgb]{0.00,0.00,0.81}{#1}}
\newcommand{\FunctionTok}[1]{\textcolor[rgb]{0.13,0.29,0.53}{\textbf{#1}}}
\newcommand{\ImportTok}[1]{#1}
\newcommand{\InformationTok}[1]{\textcolor[rgb]{0.56,0.35,0.01}{\textbf{\textit{#1}}}}
\newcommand{\KeywordTok}[1]{\textcolor[rgb]{0.13,0.29,0.53}{\textbf{#1}}}
\newcommand{\NormalTok}[1]{#1}
\newcommand{\OperatorTok}[1]{\textcolor[rgb]{0.81,0.36,0.00}{\textbf{#1}}}
\newcommand{\OtherTok}[1]{\textcolor[rgb]{0.56,0.35,0.01}{#1}}
\newcommand{\PreprocessorTok}[1]{\textcolor[rgb]{0.56,0.35,0.01}{\textit{#1}}}
\newcommand{\RegionMarkerTok}[1]{#1}
\newcommand{\SpecialCharTok}[1]{\textcolor[rgb]{0.81,0.36,0.00}{\textbf{#1}}}
\newcommand{\SpecialStringTok}[1]{\textcolor[rgb]{0.31,0.60,0.02}{#1}}
\newcommand{\StringTok}[1]{\textcolor[rgb]{0.31,0.60,0.02}{#1}}
\newcommand{\VariableTok}[1]{\textcolor[rgb]{0.00,0.00,0.00}{#1}}
\newcommand{\VerbatimStringTok}[1]{\textcolor[rgb]{0.31,0.60,0.02}{#1}}
\newcommand{\WarningTok}[1]{\textcolor[rgb]{0.56,0.35,0.01}{\textbf{\textit{#1}}}}
\usepackage{graphicx}
\makeatletter
\def\maxwidth{\ifdim\Gin@nat@width>\linewidth\linewidth\else\Gin@nat@width\fi}
\def\maxheight{\ifdim\Gin@nat@height>\textheight\textheight\else\Gin@nat@height\fi}
\makeatother
% Scale images if necessary, so that they will not overflow the page
% margins by default, and it is still possible to overwrite the defaults
% using explicit options in \includegraphics[width, height, ...]{}
\setkeys{Gin}{width=\maxwidth,height=\maxheight,keepaspectratio}
% Set default figure placement to htbp
\makeatletter
\def\fps@figure{htbp}
\makeatother
\setlength{\emergencystretch}{3em} % prevent overfull lines
\providecommand{\tightlist}{%
  \setlength{\itemsep}{0pt}\setlength{\parskip}{0pt}}
\setcounter{secnumdepth}{-\maxdimen} % remove section numbering
\ifLuaTeX
  \usepackage{selnolig}  % disable illegal ligatures
\fi
\usepackage{bookmark}
\IfFileExists{xurl.sty}{\usepackage{xurl}}{} % add URL line breaks if available
\urlstyle{same}
\hypersetup{
  pdftitle={demo},
  hidelinks,
  pdfcreator={LaTeX via pandoc}}

\title{demo}
\author{}
\date{\vspace{-2.5em}2025-04-11}

\begin{document}
\maketitle

A marginal approach to generalized linear models For
clustered/correlated data, the difficulties in interpreting the
parameter estimates from generalized linear mixed effects models with
non-identity links suggests that other approaches are needed. One of the
most popular of these is generalized estimating equations (GEE). GEE
extends generalized linear models to correlated data but differs from
mixed effects models in that GEE explicitly fits a marginal model to
data. An abstract formulation of the generalized linear model The
probability distributions used in generalized linear models are related
because they are all members of what's called the exponential family of
distributions. The density (mass) function of any member of the
exponential family takes the following form.

In Generalized Estimating Equations (GEE), the working correlation
structure refers to the assumed or approximated correlation pattern
among repeated measurements within the same subject (or observational
unit) that is specified in the estimating equations. It is termed
``working'' because the consistency of the regression coefficient
estimates is preserved in large samples even if the working correlation
structure does not match the true underlying correlation. However, if
the working structure more closely approximates the true correlation,
the estimation becomes more efficient.

Common types of working correlation structures include:
\#\#\#``independence'': the observations within the groups are
uncorrelated. \#\#\#``exchangeable'': each pair of observations in a
group has the same correlation. \#\#\#``unstructured'': each pair of
observations in a group is allowed to have a different correlation.
\#\#\#``AR-M'': this is used to fit an autoregressive structure. To
obtain a specific autoregressive structure requires the additional
argument Mv. For example corstr=``AR-M'', Mv=1 yields an AR(1)
structure, while corstr=``AR-M'', Mv=2 yields an AR(2) structure.
\#\#\#``non\_stat\_M\_dep'': stands for nonstationary M-dependent and
generates a banded correlation matrix. It also requires the Mv argument
to denote the number of nonzero off-diagonal bands that are to be
estimated. Like ``AR-M'', ``non\_stat\_M\_dep'' assumes there is a
natural order to the data. Like ``unstructured'', ``non\_stat\_M\_dep''
allows the entries within the each nonzero band to be different. As an
example corstr=``non\_stat\_M\_dep'', Mv=1 would correspond to the
following correlation matrix for a group of size 4. Here α, β, and γ are
parameters that need to be estimated. Common Methods for Selecting the
GEE Correlation Structure

There are three commonly recommended approaches for selecting a
correlation structure in GEE, as outlined by Hardin \& Hilbe (2003):

Structure Based on Data Collection Mechanism: Choose a correlation
structure that reflects how the data were collected. For example, in
longitudinal or temporal data, a structure accounting for time
dependence such as AR(1) is often appropriate.

Minimizing Pan's QIC: Select the structure that minimizes Pan's
Quasi-likelihood under the Independence model Criterion (QIC), which
generalizes the Akaike Information Criterion (AIC) for use in GEE
models. QIC is applicable only when comparing models that are identical
in all aspects except for their working correlation structures. Note:
This is distinct from QAIC as described in Burnham \& Anderson (2002).

Variance Approximation Consistency: Choose the structure for which the
sandwich (robust) estimate of the variance most closely approximates the
naïve variance estimate.

On QIC and the Issue of Misspecification

If past covariate information has additional effects on the current
response but is not sufficiently captured in the model, using
non-diagonal correlation structures (such as CS or AR-1) may introduce
bias. We will further examine how this potential misspecification
affects model selection results.

Concept of the Marginal Model

The marginal model---also referred to as a population-averaged
model---describes the average relationship between response and
covariates across the entire population. Its main features include:

No subject-specific random effects: The marginal model focuses solely on
the population-average effect. It models how the mean of the response
changes with covariates x, without accounting for individual-specific
random effects (e.g., subject-level heterogeneity).

Conclusion

A major advantage of the GEE approach is that, even in the presence of
intra-subject correlation (e.g., repeated measurements), one can obtain
consistent estimates of the regression parameters β by specifying a
working correlation structure. Importantly, GEE targets the marginal
model, which characterizes the average covariate-response relationship
at the population level, without modeling individual random effects.

In GEE, the model is typically specified only on the current covariates
x\_ij rather than attempting to incorporate historical information or
individual-specific effects. This allows for robust estimation of
population-level effects, even when repeated measures exhibit
correlation.

Assumption (2) and Its Implications

This article emphasizes the importance of Assumption (2) when using GEE.
Specifically, it requires that the expectation of the response at time
t, given the current covariates x\_it, should be consistent with the
expectation conditional on all time-point covariates.

When Assumption (2) Holds or Is Approximately Satisfied: If model
specification or covariate selection sufficiently incorporates
historical information---thus satisfying or approximately satisfying
Assumption (2)---then using a compound symmetry (CS) or AR(1) working
correlation structure is justified. These structures leverage the
correlation in repeated measures and improve estimation efficiency.
Theoretically, this leads to smaller variance in estimates and thus
greater efficiency.

When Assumption (2) Fails: If past covariates affect current outcomes
but are not adequately modeled, using non-independent working structures
such as CS or AR(1) may introduce bias. In such cases, it is preferable
to use an independence structure, which ensures unbiased parameter
estimates despite sacrificing some efficiency. The use of sandwich
(robust) variance estimation guarantees the consistency of estimates
even when correlations are ignored.

Motivation

In longitudinal or clustered data, observations within a subject or
cluster are typically correlated. Assuming independence may preserve
consistency but often results in less efficient estimates.

The GEE framework addresses this by specifying a working correlation
matrix to model intra-subject dependence and constructs the estimating
equations based on both the mean and variance structures.

If the chosen working correlation structure closely matches the true
underlying correlation, one can achieve more precise (lower variance)
estimates. However, when misspecified, although estimates remain
consistent under regularity conditions (Pepe and Anderson, 1994), they
may suffer from reduced efficiency or even bias.

Simulation Study Design

Model 1 and Model 2: Both share a marginal model of E(Y\textbar x) = βx,
but due to their dependence on past values of Y, using a non-independent
working correlation matrix (e.g., CS or AR(1)) may introduce bias. In
contrast, the independence structure tends to yield more stable
estimates.

Model 3: The marginal model is still E(Y\textbar x) = βx, but the true
data-generating correlation is compound symmetric. Hence, the CS
structure is expected to be more efficient.

To compare methods, we will perform repeated simulations (e.g., 1000
iterations) and compute the bias, variance, and mean squared error (MSE)
of β estimates under each correlation structure.

Simulation Procedure

We will proceed with several correlation structure selection methods:

First, simulate data from the three models and fit GEE models using
different working correlation structures. Calculate MSE (from empirical
variance) to identify the best-fitting structure.

Next, apply the Prediction Mean Squared Error (PMSE) approach to select
the correlation structure.

Finally, use Pan's QIC criterion to compare and select the optimal
working correlation matrix.

\begin{Shaded}
\begin{Highlighting}[]
\CommentTok{\# Simulation}
\FunctionTok{library}\NormalTok{(geepack)}

\DocumentationTok{\#\#\# The data generated for simulation}

\DocumentationTok{\#\# Model 1: Y\_it = α * Y\_i,t{-}1 + β * x\_it + ε\_it, }
\DocumentationTok{\#\# Init Y\_i0 = 0,and x\_it \textasciitilde{} N(0,1), ε\_it \textasciitilde{} N(0,1)}
\NormalTok{simulate\_model1 }\OtherTok{\textless{}{-}} \ControlFlowTok{function}\NormalTok{(N, T, alpha, beta) \{}
\NormalTok{  sim\_data }\OtherTok{\textless{}{-}} \FunctionTok{data.frame}\NormalTok{()}
  \ControlFlowTok{for}\NormalTok{ (i }\ControlFlowTok{in} \DecValTok{1}\SpecialCharTok{:}\NormalTok{N) \{}
\NormalTok{    Y\_prev }\OtherTok{\textless{}{-}} \DecValTok{0}  \CommentTok{\# }
    \ControlFlowTok{for}\NormalTok{ (t }\ControlFlowTok{in} \DecValTok{1}\SpecialCharTok{:}\NormalTok{T) \{}
\NormalTok{      x }\OtherTok{\textless{}{-}} \FunctionTok{rnorm}\NormalTok{(}\DecValTok{1}\NormalTok{, }\AttributeTok{mean =} \DecValTok{0}\NormalTok{, }\AttributeTok{sd =} \DecValTok{1}\NormalTok{)}
\NormalTok{      eps }\OtherTok{\textless{}{-}} \FunctionTok{rnorm}\NormalTok{(}\DecValTok{1}\NormalTok{, }\AttributeTok{mean =} \DecValTok{0}\NormalTok{, }\AttributeTok{sd =} \DecValTok{1}\NormalTok{)}
\NormalTok{      Y\_curr }\OtherTok{\textless{}{-}}\NormalTok{ alpha }\SpecialCharTok{*}\NormalTok{ Y\_prev }\SpecialCharTok{+}\NormalTok{ beta }\SpecialCharTok{*}\NormalTok{ x }\SpecialCharTok{+}\NormalTok{ eps}
\NormalTok{      sim\_data }\OtherTok{\textless{}{-}} \FunctionTok{rbind}\NormalTok{(sim\_data, }\FunctionTok{data.frame}\NormalTok{(}\AttributeTok{id =}\NormalTok{ i, }\AttributeTok{time =}\NormalTok{ t, }\AttributeTok{x =}\NormalTok{ x, }\AttributeTok{Y =}\NormalTok{ Y\_curr))}
\NormalTok{      Y\_prev }\OtherTok{\textless{}{-}}\NormalTok{ Y\_curr}
\NormalTok{    \}}
\NormalTok{  \}}
  \FunctionTok{return}\NormalTok{(sim\_data)}
\NormalTok{\}}

\DocumentationTok{\#\# Model 2: Y\_it = Y\_i,t{-}1 * (β * x\_it) + ε\_it, }
\DocumentationTok{\#\# Init Y\_i0 = 1,Set β = 1,x\_it \textasciitilde{} N(1,1),ε\_it \textasciitilde{} N(0,1)}
\NormalTok{simulate\_model2 }\OtherTok{\textless{}{-}} \ControlFlowTok{function}\NormalTok{(N, T, beta) \{}
\NormalTok{  sim\_data }\OtherTok{\textless{}{-}} \FunctionTok{data.frame}\NormalTok{()}
  \ControlFlowTok{for}\NormalTok{ (i }\ControlFlowTok{in} \DecValTok{1}\SpecialCharTok{:}\NormalTok{N) \{}
\NormalTok{    Y\_prev }\OtherTok{\textless{}{-}} \DecValTok{1}  \CommentTok{\# Init 1}
    \ControlFlowTok{for}\NormalTok{ (t }\ControlFlowTok{in} \DecValTok{1}\SpecialCharTok{:}\NormalTok{T) \{}
\NormalTok{      x }\OtherTok{\textless{}{-}} \FunctionTok{rnorm}\NormalTok{(}\DecValTok{1}\NormalTok{, }\AttributeTok{mean =} \DecValTok{1}\NormalTok{, }\AttributeTok{sd =} \DecValTok{1}\NormalTok{)}
\NormalTok{      eps }\OtherTok{\textless{}{-}} \FunctionTok{rnorm}\NormalTok{(}\DecValTok{1}\NormalTok{, }\AttributeTok{mean =} \DecValTok{0}\NormalTok{, }\AttributeTok{sd =} \DecValTok{1}\NormalTok{)}
\NormalTok{      Y\_curr }\OtherTok{\textless{}{-}}\NormalTok{ Y\_prev }\SpecialCharTok{*}\NormalTok{ (beta }\SpecialCharTok{*}\NormalTok{ x) }\SpecialCharTok{+}\NormalTok{ eps}
\NormalTok{      sim\_data }\OtherTok{\textless{}{-}} \FunctionTok{rbind}\NormalTok{(sim\_data, }\FunctionTok{data.frame}\NormalTok{(}\AttributeTok{id =}\NormalTok{ i, }\AttributeTok{time =}\NormalTok{ t, }\AttributeTok{x =}\NormalTok{ x, }\AttributeTok{Y =}\NormalTok{ Y\_curr))}
\NormalTok{      Y\_prev }\OtherTok{\textless{}{-}}\NormalTok{ Y\_curr}
\NormalTok{    \}}
\NormalTok{  \}}
  \FunctionTok{return}\NormalTok{(sim\_data)}
\NormalTok{\}}

\DocumentationTok{\#\# Model 3: Y\_it = b\_i + β * x\_it + ε\_it,}
\DocumentationTok{\#\# b\_i \textasciitilde{} N(0,1),x\_it \textasciitilde{} N(0,1),ε\_it \textasciitilde{} N(0,1)}
\NormalTok{simulate\_model3 }\OtherTok{\textless{}{-}} \ControlFlowTok{function}\NormalTok{(N, T, beta) \{}
\NormalTok{  sim\_data }\OtherTok{\textless{}{-}} \FunctionTok{data.frame}\NormalTok{()}
  \CommentTok{\# interception is random }
\NormalTok{  b }\OtherTok{\textless{}{-}} \FunctionTok{rnorm}\NormalTok{(N, }\AttributeTok{mean =} \DecValTok{0}\NormalTok{, }\AttributeTok{sd =} \DecValTok{1}\NormalTok{)}
  \ControlFlowTok{for}\NormalTok{ (i }\ControlFlowTok{in} \DecValTok{1}\SpecialCharTok{:}\NormalTok{N) \{}
    \ControlFlowTok{for}\NormalTok{ (t }\ControlFlowTok{in} \DecValTok{1}\SpecialCharTok{:}\NormalTok{T) \{}
\NormalTok{      x }\OtherTok{\textless{}{-}} \FunctionTok{rnorm}\NormalTok{(}\DecValTok{1}\NormalTok{, }\AttributeTok{mean =} \DecValTok{0}\NormalTok{, }\AttributeTok{sd =} \DecValTok{1}\NormalTok{)}
\NormalTok{      eps }\OtherTok{\textless{}{-}} \FunctionTok{rnorm}\NormalTok{(}\DecValTok{1}\NormalTok{, }\AttributeTok{mean =} \DecValTok{0}\NormalTok{, }\AttributeTok{sd =} \DecValTok{1}\NormalTok{)}
\NormalTok{      Y }\OtherTok{\textless{}{-}}\NormalTok{ b[i] }\SpecialCharTok{+}\NormalTok{ beta }\SpecialCharTok{*}\NormalTok{ x }\SpecialCharTok{+}\NormalTok{ eps}
\NormalTok{      sim\_data }\OtherTok{\textless{}{-}} \FunctionTok{rbind}\NormalTok{(sim\_data, }\FunctionTok{data.frame}\NormalTok{(}\AttributeTok{id =}\NormalTok{ i, }\AttributeTok{time =}\NormalTok{ t, }\AttributeTok{x =}\NormalTok{ x, }\AttributeTok{Y =}\NormalTok{ Y))}
\NormalTok{    \}}
\NormalTok{  \}}
  \FunctionTok{return}\NormalTok{(sim\_data)}
\NormalTok{\}}

\FunctionTok{set.seed}\NormalTok{(}\DecValTok{123}\NormalTok{)}
\NormalTok{N }\OtherTok{\textless{}{-}} \DecValTok{100}  
\NormalTok{T }\OtherTok{\textless{}{-}} \DecValTok{5}    \CommentTok{\#  5 observations of each individual}
\NormalTok{alpha }\OtherTok{\textless{}{-}} \FloatTok{0.5}  \CommentTok{\# Model 1 }
\NormalTok{beta\_val }\OtherTok{\textless{}{-}} \DecValTok{1}  \CommentTok{\# β }

\CommentTok{\# Generate the data for each models}
\NormalTok{data1 }\OtherTok{\textless{}{-}} \FunctionTok{simulate\_model1}\NormalTok{(N, T, alpha, beta\_val)}
\NormalTok{data2 }\OtherTok{\textless{}{-}} \FunctionTok{simulate\_model2}\NormalTok{(N, T, beta\_val)}
\NormalTok{data3 }\OtherTok{\textless{}{-}} \FunctionTok{simulate\_model3}\NormalTok{(N, T, beta\_val)}

\DocumentationTok{\#\#\# The GEE fitted models matches marginal models E(Y|x)=}
\CommentTok{\# independence, exchangeable(CS, ar1(AR(1))}

\CommentTok{\#  Model 1 }
\NormalTok{gee\_ind1 }\OtherTok{\textless{}{-}} \FunctionTok{geeglm}\NormalTok{(Y }\SpecialCharTok{\textasciitilde{}}\NormalTok{ x, }\AttributeTok{id =}\NormalTok{ id, }\AttributeTok{data =}\NormalTok{ data1, }\AttributeTok{family =}\NormalTok{ gaussian, }\AttributeTok{corstr =} \StringTok{"independence"}\NormalTok{) }\CommentTok{\#(from geepack) for fitting GEE models.}
\NormalTok{gee\_cs1  }\OtherTok{\textless{}{-}} \FunctionTok{geeglm}\NormalTok{(Y }\SpecialCharTok{\textasciitilde{}}\NormalTok{ x, }\AttributeTok{id =}\NormalTok{ id, }\AttributeTok{data =}\NormalTok{ data1, }\AttributeTok{family =}\NormalTok{ gaussian, }\AttributeTok{corstr =} \StringTok{"exchangeable"}\NormalTok{)}
\NormalTok{gee\_ar1\_1 }\OtherTok{\textless{}{-}} \FunctionTok{geeglm}\NormalTok{(Y }\SpecialCharTok{\textasciitilde{}}\NormalTok{ x, }\AttributeTok{id =}\NormalTok{ id, }\AttributeTok{data =}\NormalTok{ data1, }\AttributeTok{family =}\NormalTok{ gaussian, }\AttributeTok{corstr =} \StringTok{"ar1"}\NormalTok{)}

\CommentTok{\#  Model 2 }
\NormalTok{gee\_ind2 }\OtherTok{\textless{}{-}} \FunctionTok{geeglm}\NormalTok{(Y }\SpecialCharTok{\textasciitilde{}}\NormalTok{ x, }\AttributeTok{id =}\NormalTok{ id, }\AttributeTok{data =}\NormalTok{ data2, }\AttributeTok{family =}\NormalTok{ gaussian, }\AttributeTok{corstr =} \StringTok{"independence"}\NormalTok{)}
\NormalTok{gee\_cs2  }\OtherTok{\textless{}{-}} \FunctionTok{geeglm}\NormalTok{(Y }\SpecialCharTok{\textasciitilde{}}\NormalTok{ x, }\AttributeTok{id =}\NormalTok{ id, }\AttributeTok{data =}\NormalTok{ data2, }\AttributeTok{family =}\NormalTok{ gaussian, }\AttributeTok{corstr =} \StringTok{"exchangeable"}\NormalTok{)}
\NormalTok{gee\_ar1\_2 }\OtherTok{\textless{}{-}} \FunctionTok{geeglm}\NormalTok{(Y }\SpecialCharTok{\textasciitilde{}}\NormalTok{ x, }\AttributeTok{id =}\NormalTok{ id, }\AttributeTok{data =}\NormalTok{ data2, }\AttributeTok{family =}\NormalTok{ gaussian, }\AttributeTok{corstr =} \StringTok{"ar1"}\NormalTok{)}

\CommentTok{\#  Model 3 }
\NormalTok{gee\_ind3 }\OtherTok{\textless{}{-}} \FunctionTok{geeglm}\NormalTok{(Y }\SpecialCharTok{\textasciitilde{}}\NormalTok{ x, }\AttributeTok{id =}\NormalTok{ id, }\AttributeTok{data =}\NormalTok{ data3, }\AttributeTok{family =}\NormalTok{ gaussian, }\AttributeTok{corstr =} \StringTok{"independence"}\NormalTok{)}
\NormalTok{gee\_cs3  }\OtherTok{\textless{}{-}} \FunctionTok{geeglm}\NormalTok{(Y }\SpecialCharTok{\textasciitilde{}}\NormalTok{ x, }\AttributeTok{id =}\NormalTok{ id, }\AttributeTok{data =}\NormalTok{ data3, }\AttributeTok{family =}\NormalTok{ gaussian, }\AttributeTok{corstr =} \StringTok{"exchangeable"}\NormalTok{)}
\NormalTok{gee\_ar1\_3 }\OtherTok{\textless{}{-}} \FunctionTok{geeglm}\NormalTok{(Y }\SpecialCharTok{\textasciitilde{}}\NormalTok{ x, }\AttributeTok{id =}\NormalTok{ id, }\AttributeTok{data =}\NormalTok{ data3, }\AttributeTok{family =}\NormalTok{ gaussian, }\AttributeTok{corstr =} \StringTok{"ar1"}\NormalTok{)}


\FunctionTok{cat}\NormalTok{(}\StringTok{"Model 1 {-} GEE Estimates:}\SpecialCharTok{\textbackslash{}n}\StringTok{"}\NormalTok{)}
\end{Highlighting}
\end{Shaded}

\begin{verbatim}
## Model 1 - GEE Estimates:
\end{verbatim}

\begin{Shaded}
\begin{Highlighting}[]
\FunctionTok{print}\NormalTok{(}\FunctionTok{summary}\NormalTok{(gee\_ind1))}
\end{Highlighting}
\end{Shaded}

\begin{verbatim}
## 
## Call:
## geeglm(formula = Y ~ x, family = gaussian, data = data1, id = id, 
##     corstr = "independence")
## 
##  Coefficients:
##             Estimate  Std.err    Wald Pr(>|W|)    
## (Intercept) -0.02289  0.07093   0.104    0.747    
## x            0.94694  0.06001 248.991   <2e-16 ***
## ---
## Signif. codes:  0 '***' 0.001 '**' 0.01 '*' 0.05 '.' 0.1 ' ' 1
## 
## Correlation structure = independence 
## Estimated Scale Parameters:
## 
##             Estimate Std.err
## (Intercept)    1.337 0.08114
## Number of clusters:   100  Maximum cluster size: 5
\end{verbatim}

\begin{Shaded}
\begin{Highlighting}[]
\CommentTok{\#The for the fitted result}
\FunctionTok{print}\NormalTok{(}\FunctionTok{summary}\NormalTok{(gee\_cs1))}
\end{Highlighting}
\end{Shaded}

\begin{verbatim}
## 
## Call:
## geeglm(formula = Y ~ x, family = gaussian, data = data1, id = id, 
##     corstr = "exchangeable")
## 
##  Coefficients:
##             Estimate Std.err   Wald Pr(>|W|)    
## (Intercept)  -0.0152  0.0726   0.04     0.83    
## x             0.8481  0.0568 222.78   <2e-16 ***
## ---
## Signif. codes:  0 '***' 0.001 '**' 0.01 '*' 0.05 '.' 0.1 ' ' 1
## 
## Correlation structure = exchangeable 
## Estimated Scale Parameters:
## 
##             Estimate Std.err
## (Intercept)     1.35  0.0823
##   Link = identity 
## 
## Estimated Correlation Parameters:
##       Estimate Std.err
## alpha    0.234   0.038
## Number of clusters:   100  Maximum cluster size: 5
\end{verbatim}

\begin{Shaded}
\begin{Highlighting}[]
\FunctionTok{print}\NormalTok{(}\FunctionTok{summary}\NormalTok{(gee\_ar1\_1))}
\end{Highlighting}
\end{Shaded}

\begin{verbatim}
## 
## Call:
## geeglm(formula = Y ~ x, family = gaussian, data = data1, id = id, 
##     corstr = "ar1")
## 
##  Coefficients:
##             Estimate Std.err   Wald Pr(>|W|)    
## (Intercept)  -0.0113  0.0718   0.02     0.87    
## x             0.7820  0.0513 232.18   <2e-16 ***
## ---
## Signif. codes:  0 '***' 0.001 '**' 0.01 '*' 0.05 '.' 0.1 ' ' 1
## 
## Correlation structure = ar1 
## Estimated Scale Parameters:
## 
##             Estimate Std.err
## (Intercept)     1.36  0.0851
##   Link = identity 
## 
## Estimated Correlation Parameters:
##       Estimate Std.err
## alpha    0.439  0.0428
## Number of clusters:   100  Maximum cluster size: 5
\end{verbatim}

\begin{Shaded}
\begin{Highlighting}[]
\FunctionTok{cat}\NormalTok{(}\StringTok{"}\SpecialCharTok{\textbackslash{}n}\StringTok{Model 2 {-} GEE Estimates:}\SpecialCharTok{\textbackslash{}n}\StringTok{"}\NormalTok{)}
\end{Highlighting}
\end{Shaded}

\begin{verbatim}
## 
## Model 2 - GEE Estimates:
\end{verbatim}

\begin{Shaded}
\begin{Highlighting}[]
\FunctionTok{print}\NormalTok{(}\FunctionTok{summary}\NormalTok{(gee\_ind2))}
\end{Highlighting}
\end{Shaded}

\begin{verbatim}
## 
## Call:
## geeglm(formula = Y ~ x, family = gaussian, data = data2, id = id, 
##     corstr = "independence")
## 
##  Coefficients:
##             Estimate Std.err  Wald Pr(>|W|)    
## (Intercept)    0.474   0.121 15.38  8.8e-05 ***
## x              0.631   0.211  8.98   0.0027 ** 
## ---
## Signif. codes:  0 '***' 0.001 '**' 0.01 '*' 0.05 '.' 0.1 ' ' 1
## 
## Correlation structure = independence 
## Estimated Scale Parameters:
## 
##             Estimate Std.err
## (Intercept)     9.84    2.86
## Number of clusters:   100  Maximum cluster size: 5
\end{verbatim}

\begin{Shaded}
\begin{Highlighting}[]
\FunctionTok{print}\NormalTok{(}\FunctionTok{summary}\NormalTok{(gee\_cs2))}
\end{Highlighting}
\end{Shaded}

\begin{verbatim}
## 
## Call:
## geeglm(formula = Y ~ x, family = gaussian, data = data2, id = id, 
##     corstr = "exchangeable")
## 
##  Coefficients:
##             Estimate Std.err  Wald Pr(>|W|)    
## (Intercept)    0.693   0.174 15.79  7.1e-05 ***
## x              0.411   0.134  9.49   0.0021 ** 
## ---
## Signif. codes:  0 '***' 0.001 '**' 0.01 '*' 0.05 '.' 0.1 ' ' 1
## 
## Correlation structure = exchangeable 
## Estimated Scale Parameters:
## 
##             Estimate Std.err
## (Intercept)     9.88    2.89
##   Link = identity 
## 
## Estimated Correlation Parameters:
##       Estimate Std.err
## alpha    0.454  0.0341
## Number of clusters:   100  Maximum cluster size: 5
\end{verbatim}

\begin{Shaded}
\begin{Highlighting}[]
\FunctionTok{print}\NormalTok{(}\FunctionTok{summary}\NormalTok{(gee\_ar1\_2))}
\end{Highlighting}
\end{Shaded}

\begin{verbatim}
## 
## Call:
## geeglm(formula = Y ~ x, family = gaussian, data = data2, id = id, 
##     corstr = "ar1")
## 
##  Coefficients:
##             Estimate Std.err Wald Pr(>|W|)    
## (Intercept)    0.763   0.203 14.1  0.00018 ***
## x              0.363   0.110 10.9  0.00096 ***
## ---
## Signif. codes:  0 '***' 0.001 '**' 0.01 '*' 0.05 '.' 0.1 ' ' 1
## 
## Correlation structure = ar1 
## Estimated Scale Parameters:
## 
##             Estimate Std.err
## (Intercept)      9.9     2.9
##   Link = identity 
## 
## Estimated Correlation Parameters:
##       Estimate Std.err
## alpha    0.647  0.0303
## Number of clusters:   100  Maximum cluster size: 5
\end{verbatim}

\begin{Shaded}
\begin{Highlighting}[]
\FunctionTok{cat}\NormalTok{(}\StringTok{"}\SpecialCharTok{\textbackslash{}n}\StringTok{Model 3 {-} GEE Estimates:}\SpecialCharTok{\textbackslash{}n}\StringTok{"}\NormalTok{)}
\end{Highlighting}
\end{Shaded}

\begin{verbatim}
## 
## Model 3 - GEE Estimates:
\end{verbatim}

\begin{Shaded}
\begin{Highlighting}[]
\FunctionTok{print}\NormalTok{(}\FunctionTok{summary}\NormalTok{(gee\_ind3))}
\end{Highlighting}
\end{Shaded}

\begin{verbatim}
## 
## Call:
## geeglm(formula = Y ~ x, family = gaussian, data = data3, id = id, 
##     corstr = "independence")
## 
##  Coefficients:
##             Estimate Std.err   Wald Pr(>|W|)    
## (Intercept)  -0.1459  0.0948   2.37     0.12    
## x             0.9793  0.0657 222.02   <2e-16 ***
## ---
## Signif. codes:  0 '***' 0.001 '**' 0.01 '*' 0.05 '.' 0.1 ' ' 1
## 
## Correlation structure = independence 
## Estimated Scale Parameters:
## 
##             Estimate Std.err
## (Intercept)     1.69   0.137
## Number of clusters:   100  Maximum cluster size: 5
\end{verbatim}

\begin{Shaded}
\begin{Highlighting}[]
\FunctionTok{print}\NormalTok{(}\FunctionTok{summary}\NormalTok{(gee\_cs3))}
\end{Highlighting}
\end{Shaded}

\begin{verbatim}
## 
## Call:
## geeglm(formula = Y ~ x, family = gaussian, data = data3, id = id, 
##     corstr = "exchangeable")
## 
##  Coefficients:
##             Estimate Std.err   Wald Pr(>|W|)    
## (Intercept)   -0.146   0.095   2.36     0.12    
## x              1.014   0.056 328.44   <2e-16 ***
## ---
## Signif. codes:  0 '***' 0.001 '**' 0.01 '*' 0.05 '.' 0.1 ' ' 1
## 
## Correlation structure = exchangeable 
## Estimated Scale Parameters:
## 
##             Estimate Std.err
## (Intercept)     1.69   0.138
##   Link = identity 
## 
## Estimated Correlation Parameters:
##       Estimate Std.err
## alpha    0.417  0.0451
## Number of clusters:   100  Maximum cluster size: 5
\end{verbatim}

\begin{Shaded}
\begin{Highlighting}[]
\FunctionTok{print}\NormalTok{(}\FunctionTok{summary}\NormalTok{(gee\_ar1\_3))}
\end{Highlighting}
\end{Shaded}

\begin{verbatim}
## 
## Call:
## geeglm(formula = Y ~ x, family = gaussian, data = data3, id = id, 
##     corstr = "ar1")
## 
##  Coefficients:
##             Estimate Std.err   Wald Pr(>|W|)    
## (Intercept)  -0.1738  0.0948   3.36    0.067 .  
## x             1.0334  0.0663 242.93   <2e-16 ***
## ---
## Signif. codes:  0 '***' 0.001 '**' 0.01 '*' 0.05 '.' 0.1 ' ' 1
## 
## Correlation structure = ar1 
## Estimated Scale Parameters:
## 
##             Estimate Std.err
## (Intercept)     1.69   0.139
##   Link = identity 
## 
## Estimated Correlation Parameters:
##       Estimate Std.err
## alpha    0.624  0.0401
## Number of clusters:   100  Maximum cluster size: 5
\end{verbatim}

\begin{Shaded}
\begin{Highlighting}[]
\NormalTok{compute\_MSE }\OtherTok{\textless{}{-}} \ControlFlowTok{function}\NormalTok{(true\_beta, est) \{}
  \FunctionTok{return}\NormalTok{((est }\SpecialCharTok{{-}}\NormalTok{ true\_beta)}\SpecialCharTok{\^{}}\DecValTok{2}\NormalTok{)}
\NormalTok{\}}
\NormalTok{mse\_ind1 }\OtherTok{\textless{}{-}} \FunctionTok{compute\_MSE}\NormalTok{(beta\_val, }\FunctionTok{coef}\NormalTok{(gee\_ind1)[}\StringTok{"x"}\NormalTok{])}
\NormalTok{mse\_cs1  }\OtherTok{\textless{}{-}} \FunctionTok{compute\_MSE}\NormalTok{(beta\_val, }\FunctionTok{coef}\NormalTok{(gee\_cs1)[}\StringTok{"x"}\NormalTok{])}
\NormalTok{mse\_ar1\_1 }\OtherTok{\textless{}{-}} \FunctionTok{compute\_MSE}\NormalTok{(beta\_val, }\FunctionTok{coef}\NormalTok{(gee\_ar1\_1)[}\StringTok{"x"}\NormalTok{])}
\FunctionTok{cat}\NormalTok{(}\StringTok{"}\SpecialCharTok{\textbackslash{}n}\StringTok{Model 1: MSE for beta estimates:}\SpecialCharTok{\textbackslash{}n}\StringTok{"}\NormalTok{)}
\end{Highlighting}
\end{Shaded}

\begin{verbatim}
## 
## Model 1: MSE for beta estimates:
\end{verbatim}

\begin{Shaded}
\begin{Highlighting}[]
\FunctionTok{cat}\NormalTok{(}\StringTok{"Independence:"}\NormalTok{, mse\_ind1, }\StringTok{"}\SpecialCharTok{\textbackslash{}n}\StringTok{"}\NormalTok{)}
\end{Highlighting}
\end{Shaded}

\begin{verbatim}
## Independence: 0.00282
\end{verbatim}

\begin{Shaded}
\begin{Highlighting}[]
\FunctionTok{cat}\NormalTok{(}\StringTok{"Exchangeable:"}\NormalTok{, mse\_cs1, }\StringTok{"}\SpecialCharTok{\textbackslash{}n}\StringTok{"}\NormalTok{)}
\end{Highlighting}
\end{Shaded}

\begin{verbatim}
## Exchangeable: 0.0231
\end{verbatim}

\begin{Shaded}
\begin{Highlighting}[]
\FunctionTok{cat}\NormalTok{(}\StringTok{"AR1:"}\NormalTok{, mse\_ar1\_1, }\StringTok{"}\SpecialCharTok{\textbackslash{}n}\StringTok{"}\NormalTok{)}
\end{Highlighting}
\end{Shaded}

\begin{verbatim}
## AR1: 0.0475
\end{verbatim}

\begin{Shaded}
\begin{Highlighting}[]
\FunctionTok{library}\NormalTok{(geepack)}

\DocumentationTok{\#\# Model 1: Y\_it = α * Y\_i,t{-}1 + β * x\_it + ε\_it}
\NormalTok{simulate\_model1 }\OtherTok{\textless{}{-}} \ControlFlowTok{function}\NormalTok{(N, T, alpha, beta) \{}
\NormalTok{  sim\_data }\OtherTok{\textless{}{-}} \FunctionTok{data.frame}\NormalTok{()}
  \ControlFlowTok{for}\NormalTok{ (i }\ControlFlowTok{in} \DecValTok{1}\SpecialCharTok{:}\NormalTok{N) \{}
\NormalTok{    Y\_prev }\OtherTok{\textless{}{-}} \DecValTok{0}  \CommentTok{\# Init}
    \ControlFlowTok{for}\NormalTok{ (t }\ControlFlowTok{in} \DecValTok{1}\SpecialCharTok{:}\NormalTok{T) \{}
\NormalTok{      x }\OtherTok{\textless{}{-}} \FunctionTok{rnorm}\NormalTok{(}\DecValTok{1}\NormalTok{, }\AttributeTok{mean =} \DecValTok{0}\NormalTok{, }\AttributeTok{sd =} \DecValTok{1}\NormalTok{)}
\NormalTok{      eps }\OtherTok{\textless{}{-}} \FunctionTok{rnorm}\NormalTok{(}\DecValTok{1}\NormalTok{, }\AttributeTok{mean =} \DecValTok{0}\NormalTok{, }\AttributeTok{sd =} \DecValTok{1}\NormalTok{)}
\NormalTok{      Y\_curr }\OtherTok{\textless{}{-}}\NormalTok{ alpha }\SpecialCharTok{*}\NormalTok{ Y\_prev }\SpecialCharTok{+}\NormalTok{ beta }\SpecialCharTok{*}\NormalTok{ x }\SpecialCharTok{+}\NormalTok{ eps}
\NormalTok{      sim\_data }\OtherTok{\textless{}{-}} \FunctionTok{rbind}\NormalTok{(sim\_data, }\FunctionTok{data.frame}\NormalTok{(}\AttributeTok{id =}\NormalTok{ i, }\AttributeTok{time =}\NormalTok{ t, }\AttributeTok{x =}\NormalTok{ x, }\AttributeTok{Y =}\NormalTok{ Y\_curr))}
\NormalTok{      Y\_prev }\OtherTok{\textless{}{-}}\NormalTok{ Y\_curr}
\NormalTok{    \}}
\NormalTok{  \}}
  \FunctionTok{return}\NormalTok{(sim\_data)}
\NormalTok{\}}

\CommentTok{\#Calculate the Mean Square Error}
\NormalTok{compute\_MSE }\OtherTok{\textless{}{-}} \ControlFlowTok{function}\NormalTok{(true\_beta, est) \{}
  \FunctionTok{return}\NormalTok{((est }\SpecialCharTok{{-}}\NormalTok{ true\_beta)}\SpecialCharTok{\^{}}\DecValTok{2}\NormalTok{)}
\NormalTok{\}}

\DocumentationTok{\#\# Replications(Model 1)}
\NormalTok{simulate\_experiment }\OtherTok{\textless{}{-}} \ControlFlowTok{function}\NormalTok{(}\AttributeTok{R =} \DecValTok{100}\NormalTok{, }\AttributeTok{N =} \DecValTok{100}\NormalTok{, }\AttributeTok{T =} \DecValTok{5}\NormalTok{, }\AttributeTok{alpha =} \FloatTok{0.5}\NormalTok{, }\AttributeTok{beta\_val =} \DecValTok{1}\NormalTok{) \{}
\NormalTok{  mse\_results }\OtherTok{\textless{}{-}} \FunctionTok{matrix}\NormalTok{(}\ConstantTok{NA}\NormalTok{, }\AttributeTok{nrow =}\NormalTok{ R, }\AttributeTok{ncol =} \DecValTok{3}\NormalTok{)}
  \FunctionTok{colnames}\NormalTok{(mse\_results) }\OtherTok{\textless{}{-}} \FunctionTok{c}\NormalTok{(}\StringTok{"Independence"}\NormalTok{, }\StringTok{"Exchangeable"}\NormalTok{, }\StringTok{"AR1"}\NormalTok{)}
  
  \CommentTok{\# Data frame for counting the minimized MSE}
\NormalTok{  best\_count }\OtherTok{\textless{}{-}} \FunctionTok{c}\NormalTok{(}\AttributeTok{Independence =} \DecValTok{0}\NormalTok{, }\AttributeTok{Exchangeable =} \DecValTok{0}\NormalTok{, }\AttributeTok{AR1 =} \DecValTok{0}\NormalTok{)}
  
  \ControlFlowTok{for}\NormalTok{ (r }\ControlFlowTok{in} \DecValTok{1}\SpecialCharTok{:}\NormalTok{R) \{}

\NormalTok{    data }\OtherTok{\textless{}{-}} \FunctionTok{simulate\_model1}\NormalTok{(N, T, alpha, beta\_val)}
    
    \CommentTok{\# The same as the one reduplication simulation in previous}
\NormalTok{    gee\_ind }\OtherTok{\textless{}{-}} \FunctionTok{geeglm}\NormalTok{(Y }\SpecialCharTok{\textasciitilde{}}\NormalTok{ x, }\AttributeTok{id =}\NormalTok{ id, }\AttributeTok{data =}\NormalTok{ data, }\AttributeTok{family =}\NormalTok{ gaussian, }\AttributeTok{corstr =} \StringTok{"independence"}\NormalTok{)}
\NormalTok{    gee\_cs  }\OtherTok{\textless{}{-}} \FunctionTok{geeglm}\NormalTok{(Y }\SpecialCharTok{\textasciitilde{}}\NormalTok{ x, }\AttributeTok{id =}\NormalTok{ id, }\AttributeTok{data =}\NormalTok{ data, }\AttributeTok{family =}\NormalTok{ gaussian, }\AttributeTok{corstr =} \StringTok{"exchangeable"}\NormalTok{)}
\NormalTok{    gee\_ar1 }\OtherTok{\textless{}{-}} \FunctionTok{geeglm}\NormalTok{(Y }\SpecialCharTok{\textasciitilde{}}\NormalTok{ x, }\AttributeTok{id =}\NormalTok{ id, }\AttributeTok{data =}\NormalTok{ data, }\AttributeTok{family =}\NormalTok{ gaussian, }\AttributeTok{corstr =} \StringTok{"ar1"}\NormalTok{)}
    
    \CommentTok{\# Calculate the three β estimated 的 MSE}
\NormalTok{    mse\_ind }\OtherTok{\textless{}{-}} \FunctionTok{compute\_MSE}\NormalTok{(beta\_val, }\FunctionTok{coef}\NormalTok{(gee\_ind)[}\StringTok{"x"}\NormalTok{])}
\NormalTok{    mse\_cs  }\OtherTok{\textless{}{-}} \FunctionTok{compute\_MSE}\NormalTok{(beta\_val, }\FunctionTok{coef}\NormalTok{(gee\_cs)[}\StringTok{"x"}\NormalTok{])}
\NormalTok{    mse\_ar1 }\OtherTok{\textless{}{-}} \FunctionTok{compute\_MSE}\NormalTok{(beta\_val, }\FunctionTok{coef}\NormalTok{(gee\_ar1)[}\StringTok{"x"}\NormalTok{])}
    
\NormalTok{    mse\_results[r, ] }\OtherTok{\textless{}{-}} \FunctionTok{c}\NormalTok{(mse\_ind, mse\_cs, mse\_ar1)}
    
    \CommentTok{\# Record the structure has the minimal MSE}
\NormalTok{    current\_mse }\OtherTok{\textless{}{-}} \FunctionTok{c}\NormalTok{(mse\_ind, mse\_cs, mse\_ar1)}
\NormalTok{    best\_index }\OtherTok{\textless{}{-}} \FunctionTok{which.min}\NormalTok{(current\_mse)}
    \ControlFlowTok{if}\NormalTok{ (best\_index }\SpecialCharTok{==} \DecValTok{1}\NormalTok{) \{}
\NormalTok{      best\_count[}\StringTok{"Independence"}\NormalTok{] }\OtherTok{\textless{}{-}}\NormalTok{ best\_count[}\StringTok{"Independence"}\NormalTok{] }\SpecialCharTok{+} \DecValTok{1}
\NormalTok{    \} }\ControlFlowTok{else} \ControlFlowTok{if}\NormalTok{ (best\_index }\SpecialCharTok{==} \DecValTok{2}\NormalTok{) \{}
\NormalTok{      best\_count[}\StringTok{"Exchangeable"}\NormalTok{] }\OtherTok{\textless{}{-}}\NormalTok{ best\_count[}\StringTok{"Exchangeable"}\NormalTok{] }\SpecialCharTok{+} \DecValTok{1}
\NormalTok{    \} }\ControlFlowTok{else} \ControlFlowTok{if}\NormalTok{ (best\_index }\SpecialCharTok{==} \DecValTok{3}\NormalTok{) \{}
\NormalTok{      best\_count[}\StringTok{"AR1"}\NormalTok{] }\OtherTok{\textless{}{-}}\NormalTok{ best\_count[}\StringTok{"AR1"}\NormalTok{] }\SpecialCharTok{+} \DecValTok{1}
\NormalTok{    \}}
\NormalTok{  \}}
  
  \CommentTok{\# Mean MSE and MSE(VAR)}
\NormalTok{  mse\_means }\OtherTok{\textless{}{-}} \FunctionTok{colMeans}\NormalTok{(mse\_results)}
\NormalTok{  mse\_vars  }\OtherTok{\textless{}{-}} \FunctionTok{apply}\NormalTok{(mse\_results, }\DecValTok{2}\NormalTok{, var)}
  
  \FunctionTok{return}\NormalTok{(}\FunctionTok{list}\NormalTok{(}\AttributeTok{mse\_means =}\NormalTok{ mse\_means, }\AttributeTok{mse\_vars =}\NormalTok{ mse\_vars, }\AttributeTok{best\_count =}\NormalTok{ best\_count, }\AttributeTok{raw\_results =}\NormalTok{ mse\_results))}
\NormalTok{\}}

\CommentTok{\# 100 reduplicates, 5 Observations, set beta as 1.0}
\FunctionTok{set.seed}\NormalTok{(}\DecValTok{123}\NormalTok{)}
\NormalTok{results }\OtherTok{\textless{}{-}} \FunctionTok{simulate\_experiment}\NormalTok{(}\AttributeTok{R =} \DecValTok{100}\NormalTok{, }\AttributeTok{N =} \DecValTok{100}\NormalTok{, }\AttributeTok{T =} \DecValTok{5}\NormalTok{, }\AttributeTok{alpha =} \FloatTok{0.5}\NormalTok{, }\AttributeTok{beta\_val =} \DecValTok{1}\NormalTok{)}

\FunctionTok{cat}\NormalTok{(}\StringTok{"{-}{-}{-}{-}{-} Model 1 Simulation Results {-}{-}{-}{-}{-}}\SpecialCharTok{\textbackslash{}n}\StringTok{"}\NormalTok{)}
\end{Highlighting}
\end{Shaded}

\begin{verbatim}
## ----- Model 1 Simulation Results -----
\end{verbatim}

\begin{Shaded}
\begin{Highlighting}[]
\FunctionTok{cat}\NormalTok{(}\StringTok{"}\SpecialCharTok{\textbackslash{}n}\StringTok{Mean MSE:}\SpecialCharTok{\textbackslash{}n}\StringTok{"}\NormalTok{)}
\end{Highlighting}
\end{Shaded}

\begin{verbatim}
## 
## Mean MSE:
\end{verbatim}

\begin{Shaded}
\begin{Highlighting}[]
\FunctionTok{print}\NormalTok{(results}\SpecialCharTok{$}\NormalTok{mse\_means)}
\end{Highlighting}
\end{Shaded}

\begin{verbatim}
## Independence Exchangeable          AR1 
##      0.00242      0.01219      0.03393
\end{verbatim}

\begin{Shaded}
\begin{Highlighting}[]
\FunctionTok{cat}\NormalTok{(}\StringTok{"}\SpecialCharTok{\textbackslash{}n}\StringTok{MSE MSE(VAR):}\SpecialCharTok{\textbackslash{}n}\StringTok{"}\NormalTok{)}
\end{Highlighting}
\end{Shaded}

\begin{verbatim}
## 
## MSE MSE(VAR):
\end{verbatim}

\begin{Shaded}
\begin{Highlighting}[]
\FunctionTok{print}\NormalTok{(results}\SpecialCharTok{$}\NormalTok{mse\_vars)}
\end{Highlighting}
\end{Shaded}

\begin{verbatim}
## Independence Exchangeable          AR1 
##     7.61e-06     1.10e-04     2.68e-04
\end{verbatim}

\begin{Shaded}
\begin{Highlighting}[]
\FunctionTok{cat}\NormalTok{(}\StringTok{"}\SpecialCharTok{\textbackslash{}n}\StringTok{ Frequency of the working correlation matrix selected:}\SpecialCharTok{\textbackslash{}n}\StringTok{"}\NormalTok{)}
\end{Highlighting}
\end{Shaded}

\begin{verbatim}
## 
##  Frequency of the working correlation matrix selected:
\end{verbatim}

\begin{Shaded}
\begin{Highlighting}[]
\FunctionTok{print}\NormalTok{(results}\SpecialCharTok{$}\NormalTok{best\_count)}
\end{Highlighting}
\end{Shaded}

\begin{verbatim}
## Independence Exchangeable          AR1 
##           86           14            0
\end{verbatim}

\begin{Shaded}
\begin{Highlighting}[]
\FunctionTok{library}\NormalTok{(geepack)}

\DocumentationTok{\#\# {-}{-}{-}{-}{-}{-}{-}{-}{-}{-}{-}{-}{-}{-}{-}{-}{-}{-}{-}{-}{-}{-}{-}{-}{-}{-}{-}}
\DocumentationTok{\#\# Model 2:}
\DocumentationTok{\#\# Y\_it = Y\_i,t{-}1 * (β * x\_it) + ε\_it}
\DocumentationTok{\#\# Init Y\_i0 = 1, set β = 1 ,x\_it \textasciitilde{} N(1,1),ε\_it \textasciitilde{} N(0,1)}
\DocumentationTok{\#\# {-}{-}{-}{-}{-}{-}{-}{-}{-}{-}{-}{-}{-}{-}{-}{-}{-}{-}{-}{-}{-}{-}{-}{-}{-}{-}{-}}

\NormalTok{simulate\_model2 }\OtherTok{\textless{}{-}} \ControlFlowTok{function}\NormalTok{(N, T, beta) \{}
\NormalTok{  sim\_data }\OtherTok{\textless{}{-}} \FunctionTok{data.frame}\NormalTok{()}
  \ControlFlowTok{for}\NormalTok{ (i }\ControlFlowTok{in} \DecValTok{1}\SpecialCharTok{:}\NormalTok{N) \{}
\NormalTok{    Y\_prev }\OtherTok{\textless{}{-}} \DecValTok{1}  \CommentTok{\# Init 1}
    \ControlFlowTok{for}\NormalTok{ (t }\ControlFlowTok{in} \DecValTok{1}\SpecialCharTok{:}\NormalTok{T) \{}
\NormalTok{      x }\OtherTok{\textless{}{-}} \FunctionTok{rnorm}\NormalTok{(}\DecValTok{1}\NormalTok{, }\AttributeTok{mean =} \DecValTok{1}\NormalTok{, }\AttributeTok{sd =} \DecValTok{1}\NormalTok{)}
\NormalTok{      eps }\OtherTok{\textless{}{-}} \FunctionTok{rnorm}\NormalTok{(}\DecValTok{1}\NormalTok{, }\AttributeTok{mean =} \DecValTok{0}\NormalTok{, }\AttributeTok{sd =} \DecValTok{1}\NormalTok{)}
\NormalTok{      Y\_curr }\OtherTok{\textless{}{-}}\NormalTok{ Y\_prev }\SpecialCharTok{*}\NormalTok{ (beta }\SpecialCharTok{*}\NormalTok{ x) }\SpecialCharTok{+}\NormalTok{ eps}
\NormalTok{      sim\_data }\OtherTok{\textless{}{-}} \FunctionTok{rbind}\NormalTok{(sim\_data, }\FunctionTok{data.frame}\NormalTok{(}\AttributeTok{id =}\NormalTok{ i, }\AttributeTok{time =}\NormalTok{ t, }\AttributeTok{x =}\NormalTok{ x, }\AttributeTok{Y =}\NormalTok{ Y\_curr))}
\NormalTok{      Y\_prev }\OtherTok{\textless{}{-}}\NormalTok{ Y\_curr}
\NormalTok{    \}}
\NormalTok{  \}}
  \FunctionTok{return}\NormalTok{(sim\_data)}
\NormalTok{\}}


\NormalTok{compute\_MSE }\OtherTok{\textless{}{-}} \ControlFlowTok{function}\NormalTok{(true\_beta, est) \{}
  \FunctionTok{return}\NormalTok{((est }\SpecialCharTok{{-}}\NormalTok{ true\_beta)}\SpecialCharTok{\^{}}\DecValTok{2}\NormalTok{)}
\NormalTok{\}}

\DocumentationTok{\#\# Simulate of Model 2, R times replications}
\NormalTok{simulate\_experiment\_model2 }\OtherTok{\textless{}{-}} \ControlFlowTok{function}\NormalTok{(}\AttributeTok{R =} \DecValTok{100}\NormalTok{, }\AttributeTok{N =} \DecValTok{100}\NormalTok{, }\AttributeTok{T =} \DecValTok{5}\NormalTok{, }\AttributeTok{beta\_val =} \DecValTok{1}\NormalTok{) \{}
\NormalTok{  mse\_results }\OtherTok{\textless{}{-}} \FunctionTok{matrix}\NormalTok{(}\ConstantTok{NA}\NormalTok{, }\AttributeTok{nrow =}\NormalTok{ R, }\AttributeTok{ncol =} \DecValTok{3}\NormalTok{)}
  \FunctionTok{colnames}\NormalTok{(mse\_results) }\OtherTok{\textless{}{-}} \FunctionTok{c}\NormalTok{(}\StringTok{"Independence"}\NormalTok{, }\StringTok{"Exchangeable"}\NormalTok{, }\StringTok{"AR1"}\NormalTok{)}
  
  \CommentTok{\# Get the minimal MSE}
\NormalTok{  best\_count }\OtherTok{\textless{}{-}} \FunctionTok{c}\NormalTok{(}\AttributeTok{Independence =} \DecValTok{0}\NormalTok{, }\AttributeTok{Exchangeable =} \DecValTok{0}\NormalTok{, }\AttributeTok{AR1 =} \DecValTok{0}\NormalTok{)}
  
  \ControlFlowTok{for}\NormalTok{ (r }\ControlFlowTok{in} \DecValTok{1}\SpecialCharTok{:}\NormalTok{R) \{}
    \CommentTok{\# 1. Generate the simulation data of (Model 2)}
\NormalTok{    data }\OtherTok{\textless{}{-}} \FunctionTok{simulate\_model2}\NormalTok{(N, T, beta\_val)}
    
    \CommentTok{\# 2. fit GEE models, respectively}
\NormalTok{    gee\_ind }\OtherTok{\textless{}{-}} \FunctionTok{geeglm}\NormalTok{(Y }\SpecialCharTok{\textasciitilde{}}\NormalTok{ x, }\AttributeTok{id =}\NormalTok{ id, }\AttributeTok{data =}\NormalTok{ data, }\AttributeTok{family =}\NormalTok{ gaussian, }\AttributeTok{corstr =} \StringTok{"independence"}\NormalTok{)}
\NormalTok{    gee\_cs  }\OtherTok{\textless{}{-}} \FunctionTok{geeglm}\NormalTok{(Y }\SpecialCharTok{\textasciitilde{}}\NormalTok{ x, }\AttributeTok{id =}\NormalTok{ id, }\AttributeTok{data =}\NormalTok{ data, }\AttributeTok{family =}\NormalTok{ gaussian, }\AttributeTok{corstr =} \StringTok{"exchangeable"}\NormalTok{)}
\NormalTok{    gee\_ar1 }\OtherTok{\textless{}{-}} \FunctionTok{geeglm}\NormalTok{(Y }\SpecialCharTok{\textasciitilde{}}\NormalTok{ x, }\AttributeTok{id =}\NormalTok{ id, }\AttributeTok{data =}\NormalTok{ data, }\AttributeTok{family =}\NormalTok{ gaussian, }\AttributeTok{corstr =} \StringTok{"ar1"}\NormalTok{)}
    
    \CommentTok{\# 3. Calculate the MSE }
\NormalTok{    mse\_ind }\OtherTok{\textless{}{-}} \FunctionTok{compute\_MSE}\NormalTok{(beta\_val, }\FunctionTok{coef}\NormalTok{(gee\_ind)[}\StringTok{"x"}\NormalTok{])}
\NormalTok{    mse\_cs  }\OtherTok{\textless{}{-}} \FunctionTok{compute\_MSE}\NormalTok{(beta\_val, }\FunctionTok{coef}\NormalTok{(gee\_cs)[}\StringTok{"x"}\NormalTok{])}
\NormalTok{    mse\_ar1 }\OtherTok{\textless{}{-}} \FunctionTok{compute\_MSE}\NormalTok{(beta\_val, }\FunctionTok{coef}\NormalTok{(gee\_ar1)[}\StringTok{"x"}\NormalTok{])}
    
\NormalTok{    mse\_vec }\OtherTok{\textless{}{-}} \FunctionTok{c}\NormalTok{(mse\_ind, mse\_cs, mse\_ar1)}
\NormalTok{    mse\_results[r, ] }\OtherTok{\textless{}{-}}\NormalTok{ mse\_vec}
    
    \CommentTok{\# Record the structure has the minimal MSE}
\NormalTok{    best\_idx }\OtherTok{\textless{}{-}} \FunctionTok{which.min}\NormalTok{(mse\_vec)}
    \ControlFlowTok{if}\NormalTok{ (best\_idx }\SpecialCharTok{==} \DecValTok{1}\NormalTok{) \{}
\NormalTok{      best\_count[}\StringTok{"Independence"}\NormalTok{] }\OtherTok{\textless{}{-}}\NormalTok{ best\_count[}\StringTok{"Independence"}\NormalTok{] }\SpecialCharTok{+} \DecValTok{1}
\NormalTok{    \} }\ControlFlowTok{else} \ControlFlowTok{if}\NormalTok{ (best\_idx }\SpecialCharTok{==} \DecValTok{2}\NormalTok{) \{}
\NormalTok{      best\_count[}\StringTok{"Exchangeable"}\NormalTok{] }\OtherTok{\textless{}{-}}\NormalTok{ best\_count[}\StringTok{"Exchangeable"}\NormalTok{] }\SpecialCharTok{+} \DecValTok{1}
\NormalTok{    \} }\ControlFlowTok{else}\NormalTok{ \{}
\NormalTok{      best\_count[}\StringTok{"AR1"}\NormalTok{] }\OtherTok{\textless{}{-}}\NormalTok{ best\_count[}\StringTok{"AR1"}\NormalTok{] }\SpecialCharTok{+} \DecValTok{1}
\NormalTok{    \}}
\NormalTok{  \}}
  
  
\NormalTok{  mse\_means }\OtherTok{\textless{}{-}} \FunctionTok{colMeans}\NormalTok{(mse\_results)}
\NormalTok{  mse\_vars }\OtherTok{\textless{}{-}} \FunctionTok{apply}\NormalTok{(mse\_results, }\DecValTok{2}\NormalTok{, var)}
  
  
  \FunctionTok{return}\NormalTok{(}\FunctionTok{list}\NormalTok{(}
    \AttributeTok{mse\_means =}\NormalTok{ mse\_means,}
    \AttributeTok{mse\_vars =}\NormalTok{ mse\_vars,}
    \AttributeTok{best\_count =}\NormalTok{ best\_count,}
    \AttributeTok{raw\_results =}\NormalTok{ mse\_results}
\NormalTok{  ))}
\NormalTok{\}}

\DocumentationTok{\#\# 100 reduplicates, 5 Observations, set beta as 1.0}
\FunctionTok{set.seed}\NormalTok{(}\DecValTok{123}\NormalTok{)}
\NormalTok{results\_model2 }\OtherTok{\textless{}{-}} \FunctionTok{simulate\_experiment\_model2}\NormalTok{(}\AttributeTok{R =} \DecValTok{100}\NormalTok{, }\AttributeTok{N =} \DecValTok{100}\NormalTok{, }\AttributeTok{T =} \DecValTok{5}\NormalTok{, }\AttributeTok{beta\_val =} \DecValTok{1}\NormalTok{)}

\FunctionTok{cat}\NormalTok{(}\StringTok{"{-}{-}{-}{-}{-} Model 2 Simulation Results {-}{-}{-}{-}{-}}\SpecialCharTok{\textbackslash{}n}\StringTok{"}\NormalTok{)}
\end{Highlighting}
\end{Shaded}

\begin{verbatim}
## ----- Model 2 Simulation Results -----
\end{verbatim}

\begin{Shaded}
\begin{Highlighting}[]
\FunctionTok{cat}\NormalTok{(}\StringTok{"}\SpecialCharTok{\textbackslash{}n}\StringTok{Mean MSE:}\SpecialCharTok{\textbackslash{}n}\StringTok{"}\NormalTok{)}
\end{Highlighting}
\end{Shaded}

\begin{verbatim}
## 
## Mean MSE:
\end{verbatim}

\begin{Shaded}
\begin{Highlighting}[]
\FunctionTok{print}\NormalTok{(results\_model2}\SpecialCharTok{$}\NormalTok{mse\_means)}
\end{Highlighting}
\end{Shaded}

\begin{verbatim}
## Independence Exchangeable          AR1 
##        0.116        0.193        0.199
\end{verbatim}

\begin{Shaded}
\begin{Highlighting}[]
\FunctionTok{cat}\NormalTok{(}\StringTok{"}\SpecialCharTok{\textbackslash{}n}\StringTok{MSE MSE(VAR):}\SpecialCharTok{\textbackslash{}n}\StringTok{"}\NormalTok{)}
\end{Highlighting}
\end{Shaded}

\begin{verbatim}
## 
## MSE MSE(VAR):
\end{verbatim}

\begin{Shaded}
\begin{Highlighting}[]
\FunctionTok{print}\NormalTok{(results\_model2}\SpecialCharTok{$}\NormalTok{mse\_vars)}
\end{Highlighting}
\end{Shaded}

\begin{verbatim}
## Independence Exchangeable          AR1 
##       0.0435       0.0312       0.0252
\end{verbatim}

\begin{Shaded}
\begin{Highlighting}[]
\FunctionTok{cat}\NormalTok{(}\StringTok{"}\SpecialCharTok{\textbackslash{}n}\StringTok{ Frequency of the working correlation matrix selected:}\SpecialCharTok{\textbackslash{}n}\StringTok{"}\NormalTok{)}
\end{Highlighting}
\end{Shaded}

\begin{verbatim}
## 
##  Frequency of the working correlation matrix selected:
\end{verbatim}

\begin{Shaded}
\begin{Highlighting}[]
\FunctionTok{print}\NormalTok{(results\_model2}\SpecialCharTok{$}\NormalTok{best\_count)}
\end{Highlighting}
\end{Shaded}

\begin{verbatim}
## Independence Exchangeable          AR1 
##           71           18           11
\end{verbatim}

\begin{Shaded}
\begin{Highlighting}[]
\FunctionTok{library}\NormalTok{(geepack)}

\DocumentationTok{\#\# {-}{-}{-}{-}{-}{-}{-}{-}{-}{-}{-}{-}{-}{-}{-}{-}{-}{-}{-}{-}{-}{-}{-}{-}{-}{-}{-}}
\DocumentationTok{\#\# Model 3:}
\DocumentationTok{\#\# Y\_it = b\_i + β * x\_it + ε\_it}
\DocumentationTok{\#\# b\_i \textasciitilde{} N(0,1), x\_it \textasciitilde{} N(0,1), ε\_it \textasciitilde{} N(0,1)}
\DocumentationTok{\#\# {-}{-}{-}{-}{-}{-}{-}{-}{-}{-}{-}{-}{-}{-}{-}{-}{-}{-}{-}{-}{-}{-}{-}{-}{-}{-}{-}}

\NormalTok{simulate\_model3 }\OtherTok{\textless{}{-}} \ControlFlowTok{function}\NormalTok{(N, T, beta) \{}
\NormalTok{  sim\_data }\OtherTok{\textless{}{-}} \FunctionTok{data.frame}\NormalTok{()}
  
\NormalTok{  b }\OtherTok{\textless{}{-}} \FunctionTok{rnorm}\NormalTok{(N, }\AttributeTok{mean =} \DecValTok{0}\NormalTok{, }\AttributeTok{sd =} \DecValTok{1}\NormalTok{)}
  \ControlFlowTok{for}\NormalTok{ (i }\ControlFlowTok{in} \DecValTok{1}\SpecialCharTok{:}\NormalTok{N) \{}
    \ControlFlowTok{for}\NormalTok{ (t }\ControlFlowTok{in} \DecValTok{1}\SpecialCharTok{:}\NormalTok{T) \{}
\NormalTok{      x }\OtherTok{\textless{}{-}} \FunctionTok{rnorm}\NormalTok{(}\DecValTok{1}\NormalTok{, }\AttributeTok{mean =} \DecValTok{0}\NormalTok{, }\AttributeTok{sd =} \DecValTok{1}\NormalTok{)}
\NormalTok{      eps }\OtherTok{\textless{}{-}} \FunctionTok{rnorm}\NormalTok{(}\DecValTok{1}\NormalTok{, }\AttributeTok{mean =} \DecValTok{0}\NormalTok{, }\AttributeTok{sd =} \DecValTok{1}\NormalTok{)}
\NormalTok{      Y }\OtherTok{\textless{}{-}}\NormalTok{ b[i] }\SpecialCharTok{+}\NormalTok{ beta }\SpecialCharTok{*}\NormalTok{ x }\SpecialCharTok{+}\NormalTok{ eps}
\NormalTok{      sim\_data }\OtherTok{\textless{}{-}} \FunctionTok{rbind}\NormalTok{(sim\_data, }\FunctionTok{data.frame}\NormalTok{(}\AttributeTok{id =}\NormalTok{ i, }\AttributeTok{time =}\NormalTok{ t, }\AttributeTok{x =}\NormalTok{ x, }\AttributeTok{Y =}\NormalTok{ Y))}
\NormalTok{    \}}
\NormalTok{  \}}
  \FunctionTok{return}\NormalTok{(sim\_data)}
\NormalTok{\}}

\NormalTok{compute\_MSE }\OtherTok{\textless{}{-}} \ControlFlowTok{function}\NormalTok{(true\_beta, est) \{}
\NormalTok{  (est }\SpecialCharTok{{-}}\NormalTok{ true\_beta)}\SpecialCharTok{\^{}}\DecValTok{2}
\NormalTok{\}}

\NormalTok{simulate\_experiment\_model3 }\OtherTok{\textless{}{-}} \ControlFlowTok{function}\NormalTok{(}\AttributeTok{R =} \DecValTok{100}\NormalTok{, }\AttributeTok{N =} \DecValTok{100}\NormalTok{, }\AttributeTok{T =} \DecValTok{5}\NormalTok{, }\AttributeTok{beta\_val =} \DecValTok{1}\NormalTok{) \{}
\NormalTok{  mse\_results }\OtherTok{\textless{}{-}} \FunctionTok{matrix}\NormalTok{(}\ConstantTok{NA}\NormalTok{, }\AttributeTok{nrow =}\NormalTok{ R, }\AttributeTok{ncol =} \DecValTok{3}\NormalTok{)}
  \FunctionTok{colnames}\NormalTok{(mse\_results) }\OtherTok{\textless{}{-}} \FunctionTok{c}\NormalTok{(}\StringTok{"Independence"}\NormalTok{, }\StringTok{"Exchangeable"}\NormalTok{, }\StringTok{"AR1"}\NormalTok{)}
\NormalTok{  best\_count }\OtherTok{\textless{}{-}} \FunctionTok{c}\NormalTok{(}\AttributeTok{Independence =} \DecValTok{0}\NormalTok{, }\AttributeTok{Exchangeable =} \DecValTok{0}\NormalTok{, }\AttributeTok{AR1 =} \DecValTok{0}\NormalTok{)}
  
  \ControlFlowTok{for}\NormalTok{ (r }\ControlFlowTok{in} \DecValTok{1}\SpecialCharTok{:}\NormalTok{R) \{}
\NormalTok{    data }\OtherTok{\textless{}{-}} \FunctionTok{simulate\_model3}\NormalTok{(N, T, beta\_val)}

\NormalTok{    gee\_ind }\OtherTok{\textless{}{-}} \FunctionTok{geeglm}\NormalTok{(Y }\SpecialCharTok{\textasciitilde{}}\NormalTok{ x, }\AttributeTok{id =}\NormalTok{ id, }\AttributeTok{data =}\NormalTok{ data, }\AttributeTok{family =}\NormalTok{ gaussian, }\AttributeTok{corstr =} \StringTok{"independence"}\NormalTok{)}
\NormalTok{    gee\_cs  }\OtherTok{\textless{}{-}} \FunctionTok{geeglm}\NormalTok{(Y }\SpecialCharTok{\textasciitilde{}}\NormalTok{ x, }\AttributeTok{id =}\NormalTok{ id, }\AttributeTok{data =}\NormalTok{ data, }\AttributeTok{family =}\NormalTok{ gaussian, }\AttributeTok{corstr =} \StringTok{"exchangeable"}\NormalTok{)}
\NormalTok{    gee\_ar1 }\OtherTok{\textless{}{-}} \FunctionTok{geeglm}\NormalTok{(Y }\SpecialCharTok{\textasciitilde{}}\NormalTok{ x, }\AttributeTok{id =}\NormalTok{ id, }\AttributeTok{data =}\NormalTok{ data, }\AttributeTok{family =}\NormalTok{ gaussian, }\AttributeTok{corstr =} \StringTok{"ar1"}\NormalTok{)}

\NormalTok{    mse\_ind }\OtherTok{\textless{}{-}} \FunctionTok{compute\_MSE}\NormalTok{(beta\_val, }\FunctionTok{coef}\NormalTok{(gee\_ind)[}\StringTok{"x"}\NormalTok{])}
\NormalTok{    mse\_cs  }\OtherTok{\textless{}{-}} \FunctionTok{compute\_MSE}\NormalTok{(beta\_val, }\FunctionTok{coef}\NormalTok{(gee\_cs)[}\StringTok{"x"}\NormalTok{])}
\NormalTok{    mse\_ar1 }\OtherTok{\textless{}{-}} \FunctionTok{compute\_MSE}\NormalTok{(beta\_val, }\FunctionTok{coef}\NormalTok{(gee\_ar1)[}\StringTok{"x"}\NormalTok{])}
    
\NormalTok{    mse\_vec }\OtherTok{\textless{}{-}} \FunctionTok{c}\NormalTok{(mse\_ind, mse\_cs, mse\_ar1)}
\NormalTok{    mse\_results[r, ] }\OtherTok{\textless{}{-}}\NormalTok{ mse\_vec}
    
\NormalTok{    best\_idx }\OtherTok{\textless{}{-}} \FunctionTok{which.min}\NormalTok{(mse\_vec)}
    \ControlFlowTok{if}\NormalTok{ (best\_idx }\SpecialCharTok{==} \DecValTok{1}\NormalTok{) \{}
\NormalTok{      best\_count[}\StringTok{"Independence"}\NormalTok{] }\OtherTok{\textless{}{-}}\NormalTok{ best\_count[}\StringTok{"Independence"}\NormalTok{] }\SpecialCharTok{+} \DecValTok{1}
\NormalTok{    \} }\ControlFlowTok{else} \ControlFlowTok{if}\NormalTok{ (best\_idx }\SpecialCharTok{==} \DecValTok{2}\NormalTok{) \{}
\NormalTok{      best\_count[}\StringTok{"Exchangeable"}\NormalTok{] }\OtherTok{\textless{}{-}}\NormalTok{ best\_count[}\StringTok{"Exchangeable"}\NormalTok{] }\SpecialCharTok{+} \DecValTok{1}
\NormalTok{    \} }\ControlFlowTok{else}\NormalTok{ \{}
\NormalTok{      best\_count[}\StringTok{"AR1"}\NormalTok{] }\OtherTok{\textless{}{-}}\NormalTok{ best\_count[}\StringTok{"AR1"}\NormalTok{] }\SpecialCharTok{+} \DecValTok{1}
\NormalTok{    \}}
\NormalTok{  \}}
  
\NormalTok{  mse\_means }\OtherTok{\textless{}{-}} \FunctionTok{colMeans}\NormalTok{(mse\_results)}
\NormalTok{  mse\_vars  }\OtherTok{\textless{}{-}} \FunctionTok{apply}\NormalTok{(mse\_results, }\DecValTok{2}\NormalTok{, var)}
  
  \FunctionTok{return}\NormalTok{(}\FunctionTok{list}\NormalTok{(}
    \AttributeTok{mse\_means =}\NormalTok{ mse\_means,}
    \AttributeTok{mse\_vars =}\NormalTok{ mse\_vars,}
    \AttributeTok{best\_count =}\NormalTok{ best\_count,}
    \AttributeTok{raw\_results =}\NormalTok{ mse\_results}
\NormalTok{  ))}
\NormalTok{\}}

\FunctionTok{set.seed}\NormalTok{(}\DecValTok{123}\NormalTok{)}
\NormalTok{results\_model3 }\OtherTok{\textless{}{-}} \FunctionTok{simulate\_experiment\_model3}\NormalTok{(}\AttributeTok{R =} \DecValTok{100}\NormalTok{, }\AttributeTok{N =} \DecValTok{100}\NormalTok{, }\AttributeTok{T =} \DecValTok{5}\NormalTok{, }\AttributeTok{beta\_val =} \DecValTok{1}\NormalTok{)}

\FunctionTok{cat}\NormalTok{(}\StringTok{"{-}{-}{-}{-}{-} Model 3 Simulation Results {-}{-}{-}{-}{-}}\SpecialCharTok{\textbackslash{}n}\StringTok{"}\NormalTok{)}
\end{Highlighting}
\end{Shaded}

\begin{verbatim}
## ----- Model 3 Simulation Results -----
\end{verbatim}

\begin{Shaded}
\begin{Highlighting}[]
\FunctionTok{cat}\NormalTok{(}\StringTok{"}\SpecialCharTok{\textbackslash{}n}\StringTok{Mean MSE:}\SpecialCharTok{\textbackslash{}n}\StringTok{"}\NormalTok{)}
\end{Highlighting}
\end{Shaded}

\begin{verbatim}
## 
## Mean MSE:
\end{verbatim}

\begin{Shaded}
\begin{Highlighting}[]
\FunctionTok{print}\NormalTok{(results\_model3}\SpecialCharTok{$}\NormalTok{mse\_means)}
\end{Highlighting}
\end{Shaded}

\begin{verbatim}
## Independence Exchangeable          AR1 
##      0.00364      0.00227      0.00260
\end{verbatim}

\begin{Shaded}
\begin{Highlighting}[]
\FunctionTok{cat}\NormalTok{(}\StringTok{"}\SpecialCharTok{\textbackslash{}n}\StringTok{MSE MSE(VAR):}\SpecialCharTok{\textbackslash{}n}\StringTok{"}\NormalTok{)}
\end{Highlighting}
\end{Shaded}

\begin{verbatim}
## 
## MSE MSE(VAR):
\end{verbatim}

\begin{Shaded}
\begin{Highlighting}[]
\FunctionTok{print}\NormalTok{(results\_model3}\SpecialCharTok{$}\NormalTok{mse\_vars)}
\end{Highlighting}
\end{Shaded}

\begin{verbatim}
## Independence Exchangeable          AR1 
##     1.66e-05     6.94e-06     8.86e-06
\end{verbatim}

\begin{Shaded}
\begin{Highlighting}[]
\FunctionTok{cat}\NormalTok{(}\StringTok{"}\SpecialCharTok{\textbackslash{}n}\StringTok{ Frequency of the working correlation matrix selected }\SpecialCharTok{\textbackslash{}n}\StringTok{"}\NormalTok{)}
\end{Highlighting}
\end{Shaded}

\begin{verbatim}
## 
##  Frequency of the working correlation matrix selected
\end{verbatim}

\begin{Shaded}
\begin{Highlighting}[]
\FunctionTok{print}\NormalTok{(results\_model3}\SpecialCharTok{$}\NormalTok{best\_count)}
\end{Highlighting}
\end{Shaded}

\begin{verbatim}
## Independence Exchangeable          AR1 
##           22           40           38
\end{verbatim}

The previously mentioned PMSE (Prediction Mean Squared Error) method is
based on the following idea:

Objective: To select a working correlation structure that minimizes the
average prediction error when forecasting new data.

Approach: Use resampling techniques such as bootstrap or
cross-validation to fit GEE models under different correlation
structures (e.g., Independence, Compound Symmetry {[}CS{]}, AR-1). For
each fitted model, calculate the prediction error on validation data.
The structure that yields the smallest average prediction error is
preferred.

Conclusion: The working correlation structure with the smallest PMSE is
typically chosen, as it suggests better alignment with the data's
underlying correlation pattern and improved predictive performance.

Why Use PMSE for Selection?

Traditional Approach: Many practitioners rely on criteria like QIC
(Quasi-likelihood under the Independence model Criterion) or subjective
judgment and model comparison to select a suitable working correlation
structure.

PMSE Perspective: Pan and Connett (2002) proposed evaluating model
performance based on its ability to predict future observations---such
as using leave-one-out cross-validation or other validation schemes.
This is a common and intuitive strategy in statistics. If a working
correlation structure captures the true within-subject dependency more
effectively, it tends to produce lower prediction errors.

We will then apply BOOT, BOOT2, and BOOTCV procedures to conduct
simulations on the three models.

\end{document}
